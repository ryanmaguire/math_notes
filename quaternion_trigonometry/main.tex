%-----------------------------------LICENSE------------------------------------%
%   This file is part of math_notes.                                           %
%                                                                              %
%   math_notes is free software: you can redistribute it and/or                %
%   modify it under the terms of the GNU General Public License as             %
%   published by the Free Software Foundation, either version 3 of the         %
%   License, or (at your option) any later version.                            %
%                                                                              %
%   math_notes is distributed in the hope that it will be useful,              %
%   but WITHOUT ANY WARRANTY; without even the implied warranty of             %
%   MERCHANTABILITY or FITNESS FOR A PARTICULAR PURPOSE.  See the              %
%   GNU General Public License for more details.                               %
%                                                                              %
%   You should have received a copy of the GNU General Public License along    %
%   with math_notes.  If not, see <https://www.gnu.org/licenses/>.             %
%------------------------------------------------------------------------------%
\documentclass{article}
%-----------------------------------LICENSE------------------------------------%
%   This file is part of math_notes.                                           %
%                                                                              %
%   math_notes is free software: you can redistribute it and/or                %
%   modify it under the terms of the GNU General Public License as             %
%   published by the Free Software Foundation, either version 3 of the         %
%   License, or (at your option) any later version.                            %
%                                                                              %
%   math_notes is distributed in the hope that it will be useful,              %
%   but WITHOUT ANY WARRANTY; without even the implied warranty of             %
%   MERCHANTABILITY or FITNESS FOR A PARTICULAR PURPOSE.  See the              %
%   GNU General Public License for more details.                               %
%                                                                              %
%   You should have received a copy of the GNU General Public License along    %
%   with math_notes.  If not, see <https://www.gnu.org/licenses/>.             %
%------------------------------------------------------------------------------%
\usepackage{amssymb, amsmath, amsthm}
\usepackage{hyperref}
\hypersetup{colorlinks = true, linkcolor = blue}
\theoremstyle{plain}
\newtheorem{theorem}{Theorem}
\newtheoremstyle{normal}{\topsep}{\topsep}{}{}{\bfseries}{}{0.5em}{}
\theoremstyle{normal}
\newtheorem{examplex}{Example}
\newtheorem{definitionx}{Definition}
\newtheorem{notationx}{Notation}
\newcommand{\bqed}{\hfill$\blacksquare$}
\newenvironment{example}{\pushQED{\bqed}\examplex}{\popQED\endexamplex}
\newenvironment{definition}{\pushQED{\bqed}\definitionx}{\popQED\enddefinitionx}
\newenvironment{notation}{\pushQED{\bqed}\notationx}{\popQED\endnotationx}
\setlength{\parindent}{0em}
\setlength{\parskip}{0em}

\title{Quaternion Trigonometry}
\author{Ryan Maguire}
\date{\today}
\begin{document}
    \maketitle
    \begin{abstract}
        Computations with trigonmetric functions for real and complex inputs
        have long been studied. Efficient algorithms for these calculations
        exist and have been implemented in various programming libraries such
        as \texttt{glibc} or \texttt{openlibm}.
        For a real variable these functions can be computed by reducing the
        argmument mod $\pi$ and then invoking a Remez minimax polynomial, a
        Pade approximant, a Taylor series, a lookup table, or (most likely)
        some combination of these methods. For complex variables we invoke
        Euler's formula and reduce our work to computations with real numbers.
        For a quaternion variable these functions are still well-defined, but
        their computation requires more care due to the loss of commutativity.
        In this note we'll show how to reduce the computation of trigonometric
        functions of a quaternion variable to the evaluation of real-valued
        functions.
    \end{abstract}
    \tableofcontents
    \section{Quaternions}
        Complex numbers have long been used as a tool for understanding
        rotations in the plane. This can be seen using the polar
        representation of complex number. If we are given $z=re^{i\theta}$ and
        wish to rotate $z$ by an angle $\omega$ made with the positive real
        axis (the $x$ axis), we can define
        $w=e^{i\omega}$ and compute the product:
        \begin{equation}
            \label{eqn:complex_numbers_as_rotations}
            z\cdot{w}=re^{i\theta}\cdot{e}^{i\omega}=re^{i(\theta+\omega)}
        \end{equation}
        The resulting complex number has the same magnitude, but the angle
        has been shifted from $\theta$ to $\theta+\omega$, precisely what a
        rotation is supposed to do.
        \par\hfill\par
        In the 1830s the Irish mathematician Sir William Rowan Hamilton began
        trying to extend this idea to rotations in three dimensions
        \cite[p.~2]{BuchmannQuaternionHistory}. He thought naturally one would
        need an algebraic system of \textit{triples}, just like complex numbers
        can be thought of as an algebraic system of orderd pairs under the
        identification $z=x+iy=(x,\,y)\in\mathbb{R}^{2}$.
        After about a decade of unsuccessful attempts at creating such a
        theory, Hamilton discovered the \textit{quaternions}, an algebraic
        system of \textit{quadruplets}. Using the convention
        $1=(1,\,0,\,0,\,0)$, $i=(0,\,1,\,0,\,0)$, $j=(0,\,0,\,1,\,0)$, and
        $k=(0,\,0,\,0,\,1)$, Hamilton proposed the following rules for
        multiplication (\cite[p.~158]{HamiltonElementsOfQuaternions},
        \cite[p.~1]{HamiltonOnQuaternions}):
        \begin{equation}
            i^{2}=j^{2}=k^{2}=ijk=-1
        \end{equation}
        The product of \textit{elementary} quaternions can be computed using
        this equality. We have:
        \begin{align}
            ij&=-(ij)(k^2)=-(ijk)k=k\\
            jk&=-(i^2)(jk)=-i(ijk)=i\\
            ki&=-(ki)(j^{2})=-k(ij)j=-k^{2}j=j
        \end{align}
       But we have lost commutativity:
        \begin{align}
            ji&=j(jk)=j^{2}k=-k\\
            kj&=k(ki)=k^{2}i=-i\\
            ik&=i(ij)=i^{2}j=-j
        \end{align}
        Multiplication of arbitrary quaternions (elements of $\mathbb{R}^{4}$)
        can then be given by the distributive law. This yields a rather
        messy formula:
        \begin{align}
            (a_{1}+ib_{1}+jc_{1}+kd_{1})\cdot&(a_{2}+ib_{2}+jc_{2}+kd_{2})
            \nonumber\\
            &=(a_{1}a_{2}-b_{1}b_{2}-c_{1}c_{2}-d_{1}d_{2})\nonumber\\
            &+i(a_{1}b_{2}+a_{2}b_{1}+c_{1}d_{2}-c_{2}d_{1})\nonumber\\
            &+j(a_{1}c_{2}+a_{2}c_{1}-b_{1}d_{2}+b_{2}d_{1})\nonumber\\
            &+k(a_{1}d_{2}+a_{2}d_{1}-b_{1}c_{2}+c_{1}b_{2})
        \end{align}
        This algebraic structure on $\mathbb{R}^{4}$ is neither commutative
        (since $ij=-ji$) nor anti-commutative (since $1i=i1$). Still, it is
        associative, has an identity, and distributes over vector addition
        in $\mathbb{R}^{4}$, defined via:
        \begin{align}
            (a_{1}+ib_{1}+jc_{1}&+kd_{1})+(a_{2}+ib_{2}+jc_{2}+kd_{2})
            \nonumber\\
            &=(a_{1}+a_{2})+i(b_{1}+b_{2})+j(c_{1}+c_{2})+k(d_{1}+d_{2})
        \end{align}
        Moreover, non-zero elements have \textit{inverses}. This can be
        given in precisely the same manner as complex inverses. We define the
        quaternion conjugate via:
        \begin{equation}
            \textrm{conj}(a+ib+jc+kd)=a-ib-jc-kd
        \end{equation}
        The quaternion \textit{norm} is just the Euclidean norm in
        $\mathbb{R}^{4}$:
        \begin{equation}
            ||a+ib+jc+kd||=\sqrt{a^{2}+b^{2}+c^{2}+d^{2}}
        \end{equation}
        The multiplicative inverse is then:
        \begin{equation}
            (a+ib+jc+kd)^{-1}
            =\frac{\textrm{conj}(a+ib+jc+kd)}{||a+ib+jc+kd||^{2}}
            =\frac{a-ib-jc-kd}{a^{2}+b^{2}+c^{2}+d^{2}}
        \end{equation}
        So what we have is a \textbf{non-commutative field}, sometimes called a
        \textbf{skew-field} or a \textbf{division ring}
        \cite[p. 4]{LamNonCommutativeRings}.
        \subsection{Cayley-Dickson Construction}
            The complex numbers can be viewed as a 2-dimensional vector space
            over the real numbers endowed with a product:
            \begin{equation}
                (a,\,b)\cdot(c,\,d)=(ac-bd,\,ad+bc)
            \end{equation}
            In general an \textbf{algebra} over the real number (or any other
            field) is a vector space $V$ together with a product
            $\cdot:V\times{V}\rightarrow{V}$ that is compatible with vector
            addition and scalar multiplication. That is, the right and left
            distributive laws hold:
            \begin{align}
                \mathbf{x}\cdot(\mathbf{y}+\mathbf{z})
                &=(\mathbf{x}\cdot\mathbf{y})
                    +(\mathbf{x}\cdot\mathbf{z})\\
                (\mathbf{y}+\mathbf{z})\cdot\mathbf{x}
                &=(\mathbf{y}\cdot\mathbf{x})
                    +(\mathbf{z}\cdot\mathbf{x})
            \end{align}
            and scalar multiplication commutes and associates with the product:
            \begin{align}
                a(\mathbf{x}\cdot\mathbf{y})
                &=(a\mathbf{x})\cdot\mathbf{y}\\
                &=\mathbf{x}\cdot(a\mathbf{y})
            \end{align}
            This is quite a general definition, there is no requirement of a
            \textit{unit}, nor does the product need to obey the associative or
            commutative laws. An \textbf{associative algebra} is an algebra
            where the product satisfies the associative property:
            \begin{equation}
                (\mathbf{x}\cdot\mathbf{y})\cdot\mathbf{z}
                =\mathbf{x}\cdot(\mathbf{y}\cdot\mathbf{z})
            \end{equation}
            The lack of associativity in the definition is not without reason.
            The standard \textbf{cross product} on $\mathbb{R}^{3}$ makes
            3-dimensional Euclidean space into an algebra over the reals, but
            the cross product is not associative. For example:
            \begin{align}
                (\hat{\mathbf{x}}\times\hat{\mathbf{y}})\times\hat{\mathbf{y}}
                &=\hat{\mathbf{z}}\times\hat{\mathbf{y}}
                =-\hat{\mathbf{x}}\\
                \hat{\mathbf{x}}\times(\hat{\mathbf{y}}\times\hat{\mathbf{y}})
                &=\hat{\mathbf{x}}\times\mathbf{0}
                =\mathbf{0}
            \end{align}
            Moreover, there is no unit. For every vector
            $\mathbf{v}\in\mathbb{R}^{3}$ we have
            $\mathbf{v}\times\mathbf{v}=\mathbf{0}$. Since $\mathbf{0}$ does
            not act as a multiplicative identity, this shows the cross product
            gives us a non-unital non-associative algebra over the reals.
            \par\hfill\par
            The \textbf{Cayley-Dickson construction} gives us a way of forming
            a sequence of algebras over the reals
            (more generally, any algebra equipped with an
            \textit{involution} over any field. See
            \cite[p.~45]{SchaferNonassociativeAlgebras}).
            We start with an algebra $V_{0}$ with product $\cdot$.
            We create an algebra on $V_{1}=V_{0}\times{V}_{0}$ with product
            $\star$ as follows. For each $\mathbf{x}\in{V}_{0}$ define:
            \begin{equation}
                \overline{\mathbf{x}}=\mathbf{x}
            \end{equation}
            and for each $(\mathbf{x},\,\mathbf{y})\in{V}_{1}$ write:
            \begin{equation}
                \overline{(\mathbf{x},\,\mathbf{y})}
                =(\overline{\mathbf{x}},\,-\mathbf{y})
            \end{equation}
            We then define $\star$ via:
            \begin{equation}
                (\mathbf{x},\,\mathbf{y})\star(\mathbf{z},\,\mathbf{w})
                =(\mathbf{x}\cdot\mathbf{z}-
                    \overline{\mathbf{w}}\cdot\mathbf{y},\,
                \mathbf{w}\cdot\mathbf{x}+
                    \mathbf{y}\cdot\overline{\mathbf{z}})
            \end{equation}
            We define $V_{2}=V_{1}\times{V}_{1}$ and mimic the previous formulas
            to create a product on $V_{2}$. We can continue this recursively
            and at each step the result is an algebra over the reals with twice
            the dimension of the previous entry in the sequence.
            \par\hfill\par
            The equations become clearer if we write
            $(\mathbf{x},\,\mathbf{y})=\mathbf{x}+i\mathbf{y}$ and require
            $i^{2}=-1$. Invoking the distributive law we have:
            \begin{subequations}
                \begin{align}
                    (\mathbf{x}+i\mathbf{y})\star(\mathbf{z}+i\mathbf{w})
                    &=\mathbf{x}\cdot\mathbf{y}
                        +i\mathbf{x}\cdot\mathbf{w}
                        +i\mathbf{y}\cdot\mathbf{z}
                        +i^{2}\mathbf{x}\cdot\mathbf{w}\\
                    &=(\mathbf{x}\cdot\mathbf{z}-\mathbf{y}\cdot\mathbf{w})
                        +i(\mathbf{x}\cdot\mathbf{z}+\mathbf{y}\cdot\mathbf{z})
                \end{align}
            \end{subequations}
            If $V_{0}=\mathbb{R}$, the real numbers,
            and $\cdot$ is multiplication of real numbers,
            this construction yields the complex numbers.
            \par\hfill\par
            If $V_{0}=\mathbb{C}$, the complex numbers,
            and $\cdot$ is the multiplication of
            complex numbers, then the quaternions are $V_{1}$ in this
            construction. This gives a convenient way of dealing with
            quaternions: as ordered pairs of complex numbers:
            \begin{equation}
                a+ib+jc+kd
                =(a+ib)+(c+id)j
                =z+wj
            \end{equation}
            where we recall that $ij=k$.
            \par\hfill\par
            If we set $V_{0}=\mathbb{H}$, the quaternions,
            we can continue and study the
            \textbf{octonions} $V_{1}=\mathbb{O}$, which is an
            algebraic structure on $\mathbb{R}^{8}$, but these are
            neither commutative nor associative. Division by non-zero elements
            is still well-defined, and so the octonions (like the complex
            numbers and quaternions) are a \textbf{division algebra}
            over the reals.
            \par\hfill\par
            Repeating this with $V_{0}=\mathbb{O}$ yields the \textbf{sedenions}
            $V_{1}=\mathbb{S}$,%
            \footnote{
                The symbol $\mathbb{S}$ is often reserved for the topological
                structure on the unit circle
                $\mathbb{S}=\mathbb{S}^{1}\subset\mathbb{R}^{2}$.
            }
            a non-commutative, non-associative algebraic
            structure for $\mathbb{R}^{16}$. Here is where division is lost.
            It is possible to have \textbf{zero divisors}, non-zero sedenions
            $\mathbf{x},\mathbf{y}$ such that
            $\mathbf{x}\cdot\mathbf{y}=\mathbf{0}$. The algebraic structures
            become harder to work with as we repeat this process.
        \subsection{Power Series}
            For each $n\in\mathbb{N}$ in the Cayley-Dickson construction
            we have $V_{n}=\mathbb{R}^{2^{n}}$ as a vector space.
            Since Euclidean spaces naturally have a \textbf{norm},
            this construction yields \textbf{normed unital algebras} for each
            natural number. For the complex numbers, quaternions, and
            octonions we can use this norm to compute inverses.
            We have:
            \begin{equation}
                \mathbf{x}^{-1}
                =\frac{\textrm{conj}(\mathbf{x})}{||\mathbf{x}||^{2}}
            \end{equation}
            The norm also allows us to define a topology on the
            quaternions, which is identical to the Euclidean topology on
            $\mathbb{R}^{4}$. Once we have a topology we can study the
            convergence of sequences. In particular, we can study the
            convergence of \textbf{sequences of partial sums}. If
            $q:\mathbb{N}\rightarrow\mathbb{H}$ is a sequence of quaternions,
            and if $S:\mathbb{N}\rightarrow\mathbb{H}$ is defined by:
            \begin{equation}
                S_{N}=\sum_{n=0}^{N}q_{n}
            \end{equation}
            We can ask if there is some quaternion $p$ such that
            $q_{n}\rightarrow{p}$. This would mean for all
            $\varepsilon>0$ there is an $N\in\mathbb{N}$ such that for all
            $n\in\mathbb{N}$ with $n>N$ we have
            $||q_{n}-p||<\varepsilon$. This is the
            \textbf{metric space definition of convergence}, which is natural
            since normed vector spaces are metric spaces.
            \par\hfill\par
            Like in the case of real and complex variables, certain power
            series can be shown to converge using the
            \textbf{absolute convergence test}. For example, consider the
            exponential function:
            \begin{equation}
                \exp(q)
                =\sum_{n=0}^{\infty}\frac{q^{n}}{n!}
            \end{equation}
            This can be shown to converge for all $q\in\mathbb{H}$ since
            the sequence of partial sums:
            \begin{equation}
                S_{N}(q)=\sum_{n=0}^{N}\frac{||q||^{n}}{n!}
            \end{equation}
            is a convergent sequence of real numbers (the norm of a quaternion
            is a real number) so the original sum is absolutely convergent,
            and hence $\exp(q)$ is well-defined. Similarly the sine and cosine
            of quaternions can be defined:
            \begin{align}
                \cos(q)&=\sum_{n=0}^{\infty}(-1)^{n}\frac{q^{2n}}{(2n)!}\\
                \sin(q)&=\sum_{n=0}^{\infty}(-1)^{n}\frac{q^{2n+1}}{(2n+1)!}
            \end{align}
            Even hyperbolic trigonometric functions are well-defined:
            \begin{align}
                \cosh(q)&=\sum_{n=0}^{\infty}\frac{q^{2n}}{(2n)!}\\
                \sinh(q)&=\sum_{n=0}^{\infty}\frac{q^{2n+1}}{(2n+1)!}
            \end{align}
            But one must take care to not invoke formulas that are true for
            real-valued trigonometric and exponential functions. A classic
            example is the sum rule for exponentials:
            \begin{equation}
                \exp(x+y)=\exp(x)\exp(y)
            \end{equation}
            The proof of this uses the fact that real numbers commute under
            multiplication. In general this is all that is needed, so the
            formula holds for complex numbers as well:
            \begin{equation}
                \exp(z+w)=\exp(z)\exp(w)
            \end{equation}
            The proof is an application of the Cauchy product of series:
            \begin{equation}
                \Big(\sum_{n=0}^{\infty}a_{n}\Big)
                    \Big(\sum_{k=0}^{\infty}b_{k}\Big)
                =\sum_{n=0}^{\infty}\sum_{k=0}^{n}a_{k}b_{n-k}
            \end{equation}
            and the binomial theorem:
            \begin{equation}
                (x+y)^{n}
                =\sum_{k=0}^{n}\binom{n}{k}x^{k}y^{n-k}
                =\sum_{k=0}^{n}\frac{n!}{k!(n-k)!}x^{k}y^{n-k}
            \end{equation}
            For $\exp$ we have:
            \begin{subequations}
                \begin{align}
                    \exp(x+y)
                    &=\sum_{n=0}^{\infty}\frac{(x+y)^{n}}{n!}\\
                    &=\sum_{n=0}^{\infty}\frac{1}{n!}
                        \sum_{k=0}^{n}\binom{n}{k}x^{k}y^{n-k}\\
                    &=\sum_{n=0}^{\infty}\frac{1}{n!}
                        \sum_{k=0}^{n}\frac{n!}{k!(n-k)!}x^{k}y^{n-k}\\
                    &=\sum_{n=0}^{\infty}\sum_{k=0}^{n}
                        \frac{x^{k}}{k!}\frac{y^{n-k}}{(n-k)!}\\
                    &=\Big(\sum_{n=0}^{\infty}\frac{x^{n}}{n!}\Big)
                        \Big(\sum_{k=0}^{\infty}\frac{y^{k}}{k!}\Big)\\
                    &=\exp(x)\exp(y)
                \end{align}
            \end{subequations}
            \textbf{Mertens' theorem} tells us that the Cauchy product formula
            is valid for absolutely convergent series in any
            \textbf{Banach algebra} (a normed algebra where the norm induces
            a \textbf{complete} metric on the space), so in particular it holds
            for all $V_{n}$ in our Cayley-Hamilton construction
            (all of the Euclidean spaces are complete with their Euclidean
            norms). The binomial theorem, on the other hand, requires that the
            variables commute. We'll need to handle these power series with
            care if we wish to devise efficient means of computations.
        \subsection{Evaluation of the Exponential Function}
            The real-valued quaternions commute with all other quaternions.
            Moreover, these are the only quaternions with this property
            (the real numbers form the \textbf{center} of the quaternions).
            Since multiplication with real numbers does commute, for any
            quaternion $q\in\mathbb{H}$ and any real number
            $a\in\mathbb{R}$ we have:
            \begin{equation}
                \exp(a+q)=\exp(a)\exp(q)
            \end{equation}
            We can use this to simplify our computations. We split a
            quaternion $q$ into a \textit{real part} and a \textit{vector part},
            $q=a+\mathbf{v}=a+ib+jc+kd$. The vector part can also be thought of
            as the \textit{imaginary} or \textit{unreal} part. It is an element
            of $\mathbb{R}^{3}$. For a purely imaginary quaternion
            $\mathbf{v}$, the formula for multiplication yields a very nice
            equation for powers of $\mathbf{v}$. We have:
            \begin{subequations}
                \begin{align}
                    \mathbf{v}^{2}
                    &=(ib+jc+kd)^{2}\\
                    &=-b^{2}-c^{2}-d^{2}\\
                    &=-||\mathbf{v}||^{2}
                \end{align}
            \end{subequations}
            In particular, for even integers $n=2k$ we have:
            \begin{equation}
                \mathbf{v}^{2k}
                =(\mathbf{v}^{2})^{k}
                =(-||\mathbf{v}||^{2})^{k}
                =(-1)^{k}||\mathbf{v}||^{2k}
            \end{equation}
            which is a real number. The odd powers can be computed from this:
            \begin{equation}
                \mathbf{v}^{2k+1}
                =\mathbf{v}^{2k}\mathbf{v}
                =(-1)^{k}||\mathbf{v}||^{2k}\mathbf{v}
            \end{equation}
            and this is a strictly imaginary quaternion.
            The exponential series can then be split into even and odd parts.
            \begin{subequations}
                \begin{align}
                    \exp(\mathbf{v})
                    &=\sum_{n=0}^{\infty}\frac{\mathbf{v}^{n}}{n!}\\
                    &=\sum_{k=0}^{\infty}\frac{\mathbf{v}^{2k}}{(2k)!}
                        +\sum_{k=0}^{\infty}\frac{\mathbf{v}^{2k+1}}{(2k+1)!}\\
                    &=\sum_{k=0}^{\infty}
                        (-1)^{k}\frac{||\mathbf{v}||^{2k}}{(2k)!}
                        +\mathbf{v}
                        \sum_{k=0}^{\infty}(-1)^{k}
                            \frac{||\mathbf{v}||^{2k}}{(2k+1)}\\
                    &=\sum_{k=0}^{\infty}
                        (-1)^{k}\frac{||\mathbf{v}||^{2k}}{(2k)!}
                        +\frac{\mathbf{v}}{||\mathbf{v}||}
                        \sum_{k=0}^{\infty}(-1)^{k}
                            \frac{||\mathbf{v}||^{2k+1}}{(2k+1)}\\
                    &=\cos(||\mathbf{v}||)+
                        \frac{\mathbf{v}}{||\mathbf{v}||}\sin(||\mathbf{v}||)
                \end{align}
            \end{subequations}
            Since multiplication with real numbers commutes, for a
            quaternion $q=a+\mathbf{v}$, with $a\in\mathbb{R}$ and
            $\mathbf{v}$ strictly imaginary, we have:
            \begin{subequations}
                \label{eqn:exp_quaternion_formula}
                \begin{align}
                    \exp(a+\mathbf{v})
                    &=\exp(a)\exp(\mathbf{v})\\
                    &=\exp(a)\Big(
                        \cos(||\mathbf{v}||)+
                        \frac{\mathbf{v}}{||\mathbf{v}||}
                        \sin(||\mathbf{v}||)
                    \Big)
                \end{align}
            \end{subequations}
            Two things to note. First, we have reduced the computation of the
            exponential of a quaternion variable to the evaluation of
            exponential and trigonometric functions of a real variable.
            Second, this formula closely mimics Euler's formula for complex
            numbers. Indeed, Euler's formula is contained within it. If
            $q=a+ib$ is a \textit{purely complex} quaternion, then:
            \begin{subequations}
                \label{eqn:exp_quaternion_purely_complex}
                \begin{align}
                    \exp(a+ib)
                    &=\exp(a)\exp(ib)\\
                    &=\exp(a)\Big(
                        \cos(||ib||)+\frac{ib}{||ib||}\sin(||ib||)
                    \Big)\\
                    &=\exp(a)\Big(\cos(b)+i\textrm{sgn}(b)\sin(|b|)\Big)\\
                    &=\exp(a)\Big(\cos(b)+i\sin(b)\Big)
                \end{align}
            \end{subequations}
            where we have used the $\cos$ is an \textbf{even function} and
            $\sin$ is an \textbf{odd function} in our simplification, and
            where $\textrm{sgn}$ denotes the \textbf{sign} function.
            \par\hfill\par
            The generalization has a geometric interpretation as well.
            $\mathbf{v}/||\mathbf{v}||$ is a unit vector on the unit sphere
            $\mathbb{S}^{2}\subseteq\mathbb{R}^{3}$, the \textit{direction} of
            the vector $\mathbf{v}\in\mathbb{R}^{3}$. For the complex case,
            $ib/|b|$ is either \textit{north} or \textit{south}, for non-zero
            $b$. Topologically this is the \textbf{zero-sphere},
            $\mathbb{S}^{0}$, two isolated points.
    \section{Trigonometry with Quaternion Variables}
        Now that the exponential function has been dealt with, we turn our
        attention to the three main trigonometric functions: cosine, sine, and
        tangent. Before doing so, let's examine how these functions work when
        the variables commute. This will give us several familiar formulas for
        real-valued trigonometric functions, and since real numbers commute
        with quaternions, we'll be able to apply these formulas in the more
        general case with little modification.
        \subsection{The Angle Sum Formulas}
            Euler's formula is a special case of
            Eqn.~\ref{eqn:exp_quaternion_formula}, as shown by
            Eqn.~\ref{eqn:exp_quaternion_purely_complex}. The derivation of
            Euler's formula requires commutativity and that $i^{2}=-1$. If
            $q=a+\mathbf{v}$ is a quaternion where $\mathbf{v}$ is non-zero,
            we can write $q=a+||\mathbf{v}||\hat{\mathbf{v}}$ where
            $\hat{\mathbf{v}}$ is the unit vector in the direction of
            $\mathbf{v}$:
            \begin{equation}
                \hat{\mathbf{v}}=\frac{\mathbf{v}}{||\mathbf{v}||}
            \end{equation}
            $\hat{\mathbf{v}}$ then acts like $i$ for complex numbers: it
            squares to minus one
            \begin{equation}
                \hat{\mathbf{v}}^{2}
                =\Big(\frac{\mathbf{v}}{||\mathbf{v}||}\Big)^{2}
                =\frac{\mathbf{v}^{2}}{||\mathbf{v}||^{2}}
                =\frac{-||\mathbf{v}||^{2}}{||\mathbf{v}||^{2}}
                =-1
            \end{equation}
            But rather than taking on a zero dimensional role
            ($i$ is a single point),
            it takes on a two dimensional role ($\hat{\mathbf{v}}$ can be
            any point on the unit sphere $\mathbb{S}^{2}$). Let us now try and
            derive the angle-sum formulas for the sine and cosine of
            quaternions. For real numbers we have:
            \begin{align}
                \label{eqn:angle_sum_formula_cos}
                \cos(a+b)&=\cos(a)\cos(b)-\sin(a)\sin(b)\\
                \label{eqn:angle_sum_formula_sin}
                \sin(a+b)&=\cos(a)\sin(b)+\sin(a)\cos(b)
            \end{align}
            These can be derived using Euler's formula. We have:
            \begin{align}
                \cos(a+b)+i\sin(a+b)
                &=\exp\big(i(a+b)\big)\\
                &=\exp(ia)\exp(ib)\\
                &=\Big(\cos(a)+i\sin(a)\Big)
                    \Big(\cos(b)+i\sin(b)\Big)\\
                &=\Big(\cos(a)\cos(b)-\sin(a)\sin(b)\Big)\nonumber\\
                &\hspace{2em}
                    +i\Big(\cos(a)\sin(b)+\sin(a)\cos(b)\Big)
            \end{align}
            By comparing real and imaginary parts we obtain
            Eqns.~\ref{eqn:angle_sum_formula_cos} and
            \ref{eqn:angle_sum_formula_sin}. If $a$ and $b$ are complex
            values the formula still holes. This is because the derivation
            of $\exp(i\theta)=\cos(\theta)+i\sin(\theta)$ does not really
            require that $\theta$ be real, only that
            $(i\theta)^{n}=i^{n}\theta^{n}$. In other words, it requires that
            $i$ and $\theta$ commute. The complex numbers have no issues with
            commutativity, so we may write:
            \begin{align}
                \label{eqn:angle_sum_formula_complex_cos}
                \cos(z+w)&=\cos(z)\cos(w)-\sin(z)\sin(w)\\
                \label{eqn:angle_sum_formula_complex_sin}
                \sin(z+w)&=\cos(z)\sin(w)+\sin(z)\cos(w)
            \end{align}
            For complex numbers $z,w\in\mathbb{C}$. For quaternions it is not
            immediate that our formulas should hold. Let $q\in\mathbb{H}$ be
            a quaternion, and suppose $\mathbf{u}\in\mathbb{H}$ is a unit
            imaginary quaternion. That is,
            $\mathbf{u}=ib+jc+kd$ for some $b,c,d\in\mathbb{R}$ and
            $||\mathbf{u}||=1$. Furthermore, suppose $\mathbf{u}$ commutes with
            $q$, $\mathbf{u}q=q\mathbf{u}$. If this is true, then by induction
            we have $(\mathbf{u}q)^{n}=\mathbf{u}^{n}q^{n}$ for all
            $n\in\mathbb{N}$. Using this we obtain:
            \begin{subequations}
                \begin{align}
                    \exp(\mathbf{u}q)
                    &=\sum_{n=0}^{\infty}\frac{(\mathbf{u}q)^{n}}{n!}\\
                    &=\sum_{n=0}^{\infty}\mathbf{u}^{n}\frac{q^{n}}{n!}\\
                    &=\sum_{n=0}^{\infty}\mathbf{u}^{2n}\frac{q^{2n}}{(2n)!}
                        +\sum_{n=0}^{\infty}
                        \mathbf{u}^{2n+1}\frac{q^{2n+1}}{(2n+1)!}\\
                    &=\sum_{n=0}^{\infty}(-1)^{n}\frac{q^{2n}}{(2n)!}
                        +\mathbf{u}
                        \sum_{n=0}^{\infty}(-1)^{n}\frac{q^{2n+1}}{(2n+1)!}\\
                    &=\cos(q)+\mathbf{u}\sin(q)
                \end{align}
            \end{subequations}
            A direct mimicry of Euler's formula. If
            $q=a+\mathbf{v}$, where $\mathbf{v}$ is non-zero,
            then $\mathbf{u}=\hat{\mathbf{v}}$ is a unit vector that commutes
            with $q$ since:
            \begin{equation}
                \hat{\mathbf{v}}q
                =\hat{\mathbf{v}}(a+\mathbf{v})
                =\hat{\mathbf{v}}a+\hat{\mathbf{v}}\mathbf{v}
                =a\hat{\mathbf{v}}+\mathbf{v}\hat{\mathbf{v}}
                =(a+\mathbf{v})\hat{\mathbf{v}}
                =q\hat{\mathbf{v}}
            \end{equation}
            Since $\hat{\mathbf{v}}a=a\hat{\mathbf{v}}$ and
            $\hat{\mathbf{v}}\mathbf{v}=\mathbf{v}\hat{\mathbf{v}}$
            we have:
            \begin{equation}
                \exp\big(\hat{\mathbf{v}}(a+\mathbf{v})\big)=
                \exp(\hat{\mathbf{v}}a)\exp(\hat{\mathbf{v}}\mathbf{v})
            \end{equation}
            Using this we can string together the following equalities.
            \begin{subequations}
                \begin{align}
                    \cos(a+\mathbf{v})+\hat{\mathbf{v}}\sin(a+\mathbf{v})
                    &=\exp\big(\hat{\mathbf{v}}(a+\mathbf{v})\big)\\
                    &=\exp(\hat{\mathbf{v}}a)\exp(\hat{\mathbf{v}}\mathbf{v})\\
                    &=\Big(\cos(a)+\hat{\mathbf{v}}\sin(a)\Big)\Big(
                            \cos(\mathbf{v})+\hat{\mathbf{v}}\sin(\mathbf{v})
                        \Big)\\
                    &=\Big(\cos(a)\cos(\mathbf{v})-\sin(a)\sin(\mathbf{v})\Big)
                        \nonumber\\
                        &\hspace{2em}
                        +\hat{\mathbf{v}}\Big(
                            \cos(a)\sin(\mathbf{v})+\sin(a)\cos(\mathbf{v})
                        \Big)
                \end{align}
            \end{subequations}
            It would be nice to say
            \textit{compare the real and imaginary parts}, just as we did with
            Euler's formula for complex numbers, but
            $\cos(a+\mathbf{v})$ and
            $\cos(a)\cos(\mathbf{v})-\sin(a)\sin(\mathbf{v})$ may have both
            real and imaginary components. Similarly,
            $\sin(a+\mathbf{v})$ and
            $\sin(a)\cos(\mathbf{v})+\cos(a)\sin(\mathbf{v})$ may have mixed
            parts, so we can't guarantee equality here. However,
            repeating this calculation with $\mathbf{u}=-\hat{\mathbf{v}}$,
            which still commutes with $q=a+\mathbf{v}$, shows that we can cancel
            to obtain:
            \begin{align}
                \cos(a+\mathbf{v})
                &=\cos(a)\cos(\mathbf{v})-\sin(a)\sin(\mathbf{v})\\
                \sin(a+\mathbf{v})
                &=\cos(a)\sin(\mathbf{v})+\sin(a)\cos(\mathbf{v})
            \end{align}
            And with this we're one step closer to obtain general formulas.
        \subsection{Sine, Cosine, and Tangent for Quaternions}
            Now that we have the angle sum formulas for sine and cosine, the
            computation of trigonometric functions for general quaternions comes
            with ease. For a strictly imaginary $\mathbf{v}\in\mathbb{H}$
            we have:
            \begin{subequations}
                \begin{align}
                    \cos(\mathbf{v})
                    &=\sum_{n=0}^{\infty}(-1)^{n}\frac{\mathbf{v}^{2n}}{(2n)!}\\
                    &=\sum_{n=0}^{\infty}(-1)^{n}
                        \frac{(-1)^{n}||\mathbf{v}||^{2n}}{(2n)!}\\
                    &=\sum_{n=0}^{\infty}\frac{||\mathbf{v}||^{2n}}{(2n)!}\\
                    &=\cosh(||\mathbf{v}||)
                \end{align}
            \end{subequations}
            Similarly:
            \begin{subequations}
                \begin{align}
                    \sin(\mathbf{v})
                    &=\sum_{n=0}^{\infty}(-1)^{n}
                        \frac{\mathbf{v}^{2n+1}}{(2n+1)!}\\
                    &=\sum_{n=0}^{\infty}(-1)^{n}
                        \frac{(-1)^{n}\mathbf{v}||\mathbf{v}||^{2n}}{(2n+1)!}\\
                    &=\mathbf{v}\sum_{n=0}^{\infty}
                        \frac{||\mathbf{v}||^{2n}}{(2n+1)}\\
                    &=\frac{\mathbf{v}}{||\mathbf{v}||}
                        \sum_{n=0}^{\infty}
                        \frac{||\mathbf{v}||^{2n+1}}{(2n+1)!}\\
                    &=\frac{\mathbf{v}}{||\mathbf{v}||}\sinh(||\mathbf{v}||)
                \end{align}
            \end{subequations}
            And so for strictly imaginary quaternions sine and cosine have been
            reduced to hyperbolic sine and hyperbolic cosine of a real input.
            For the general quaternion we invoke the angle sum formula. If
            $q=a+\mathbf{v}$ we have:
            \begin{subequations}
                \begin{align}
                    \cos(q)
                    &=\cos(a+\mathbf{v})\\
                    &=\cos(a)\cos(\mathbf{v})-\sin(a)\sin(\mathbf{v})\\
                    &=\cos(a)\cosh(||\mathbf{v}||)
                        -\frac{\mathbf{v}}{||\mathbf{v}||}
                        \sin(a)\sinh(||\mathbf{v}||)
                \end{align}
            \end{subequations}
            And similarly:
            \begin{subequations}
                \begin{align}
                    \sin(q)
                    &=\sin(a+\mathbf{v})\\
                    &=\cos(a)\sin(\mathbf{v})+\sin(a)\cos(\mathbf{v})\\
                    &=\frac{\mathbf{v}}{||\mathbf{v}||}
                        \cos(a)\sinh(||\mathbf{v}||)
                        +\sin(a)\cosh(||\mathbf{v}||)\\
                    &=\sin(a)\cosh(||\mathbf{v}||)+
                        \frac{\mathbf{v}}{||\mathbf{v}||}
                        \cos(a)\sinh(||\mathbf{v}||)
                \end{align}
            \end{subequations}
    \section{Implementation in \texttt{libtmpl}}
    \newpage
    \bibliographystyle{plain}
    \bibliography{bib.bib}
\end{document}
