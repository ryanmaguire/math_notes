%-----------------------------------LICENSE------------------------------------%
%   This file is part of math_notes.                                           %
%                                                                              %
%   math_notes is free software: you can redistribute it and/or                %
%   modify it under the terms of the GNU General Public License as             %
%   published by the Free Software Foundation, either version 3 of the         %
%   License, or (at your option) any later version.                            %
%                                                                              %
%   math_notes is distributed in the hope that it will be useful,              %
%   but WITHOUT ANY WARRANTY; without even the implied warranty of             %
%   MERCHANTABILITY or FITNESS FOR A PARTICULAR PURPOSE.  See the              %
%   GNU General Public License for more details.                               %
%                                                                              %
%   You should have received a copy of the GNU General Public License along    %
%   with math_notes.  If not, see <https://www.gnu.org/licenses/>.             %
%------------------------------------------------------------------------------%
\usepackage{amssymb, amsmath, amsthm}
\usepackage{hyperref}
\hypersetup{colorlinks = true, linkcolor = blue}
\theoremstyle{plain}
\newtheorem{theorem}{Theorem}
\newtheoremstyle{normal}{\topsep}{\topsep}{}{}{\bfseries}{}{0.5em}{}
\theoremstyle{normal}
\newtheorem{examplex}{Example}
\newtheorem{definitionx}{Definition}
\newtheorem{notationx}{Notation}
\newcommand{\bqed}{\hfill$\blacksquare$}
\newenvironment{example}{\pushQED{\bqed}\examplex}{\popQED\endexamplex}
\newenvironment{definition}{\pushQED{\bqed}\definitionx}{\popQED\enddefinitionx}
\newenvironment{notation}{\pushQED{\bqed}\notationx}{\popQED\endnotationx}
\setlength{\parindent}{0em}
\setlength{\parskip}{0em}
